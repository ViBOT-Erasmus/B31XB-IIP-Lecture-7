%%%%%%%%%%%%%%%%%%%%%%%%%%%%%%%%%%%%% 
%% LE2I beamer template
%% Guillaume Lemaitre, October 2014
%%%%%%%%%%%%%%%%%%%%%%%%%%%%%%%%%%%%% 

\documentclass{beamer}

\usepackage[utf8]{inputenc}
\usepackage[T1]{fontenc} 
\usetheme{le2i} 

%% The amssymb package provides various useful mathematical symbols
\usepackage{amssymb}
%% The amsthm package provides extended theorem environments
\usepackage{amsthm}

%% amsmath for math environment
\usepackage{amsmath}

\DeclareMathOperator*{\argmin}{arg\,min}
\DeclareMathOperator*{\argmax}{arg\,max}
\DeclareMathOperator*{\sign}{sign}

%% figure package
\usepackage{epsf,graphicx}
\usepackage{epstopdf}
%\usepackage{subfigure}
\usepackage{subfig}
\usepackage{transparent}

%% In order to draw some graphs
\usepackage{tikz,xifthen}
\usepackage{tikz-qtree}
\usepackage{adjustbox}
\usetikzlibrary{decorations.pathmorphing}
\usetikzlibrary{fit}
\usetikzlibrary{backgrounds}
\usetikzlibrary{shapes,arrows,shadows}
\usetikzlibrary{calc,decorations.pathreplacing,decorations.markings,positioning}
\usetikzlibrary{snakes,decorations.text,shapes,patterns}
% \usepackage{scalefnt,lmodern,booktabs}

%% Package for cross and tick symbols
\usepackage{pifont}
\newcommand{\tick}{\color{green!60!black!80}\ding{51}}
\newcommand{\cross}{\color{red!60!black!80}\ding{55}}

\title{Introduction to Image Processing}
\author{Guillaume Lema\^itre \\ \texttt{guillaume.lemaitre@udg.edu}}
\date{Lecture 6 \\ 2\textsuperscript{nd} Nov. 2015}

\institute{Universit\'e de Bourgogne} 

%% Uncomment if you want to avoid thousand of bullet inside the menu
% \usepackage{etoolbox}
% \makeatletter
% \patchcmd{\slideentry}{\advance\beamer@xpos by1\relax}{}{}{}
% \def\beamer@subsectionentry#1#2#3#4#5{\advance\beamer@xpos by1\relax}%
% \makeatother

\begin{document}

% Show the title page
\begin{frame}
  \titlepage
\end{frame}

% Show the table of contents
\begin{frame}
  \tableofcontents[sectionstyle=show,subsectionstyle=show,subsubsectionstyle=hide]
\end{frame}

%---------------------

\section{Preliminaries}

\begin{frame}
\frametitle{Morphological Image Processing}
\framesubtitle{Preliminaries}
\begin{block}{Problem statement}\scriptsize
  \begin{center}
    \includegraphics[height=0.4\textheight]{images/dilation.png}
  \end{center}
    \begin{itemize}
    \item Enhance binary images
  \end{itemize}
\end{block}
\end{frame}

\begin{frame}
\frametitle{Morphological Image Processing}
\framesubtitle{Preliminaries}
\begin{block}{Structuring elements}\scriptsize
  \begin{center}
    \includegraphics[height=0.2\textheight]{images/se.png}
  \end{center}
    \begin{itemize}
    \item Affect an label value to a given pixel inside a sub-region 
    \item Consider only the shaded pixels
  \end{itemize}
\end{block}
\begin{block}{What types?}\scriptsize
  \begin{itemize}
    \item Refer to \texttt{skimage.morphology} module 
    \item ball, cube, diamond, disk, octagon, octahedron, square, star
  \end{itemize}
\end{block}
\end{frame}

\section{Erosion \& Dilation}

\subsection{Erosion}

\begin{frame}
\frametitle{Morphological Image Processing}
%\framesubtitle{Erosion}
\begin{block}{Erosion}\scriptsize
  \begin{center}
    \includegraphics[height=0.2\textheight]{images/erosion.png}
    $$A \ominus B = \lbrace z |(B)_z \subseteq A \rbrace$$
  \end{center}
  \begin{itemize}
    \item Shrink or thins objects
  \end{itemize}
\end{block}
\end{frame}

\subsection{Dilation}

\begin{frame}
\frametitle{Morphological Image Processing}
%\framesubtitle{Erosion}
\begin{block}{Dilation}\scriptsize
  \begin{center}
    \includegraphics[height=0.2\textheight]{images/dilation.png}
    $$A \oplus B = \lbrace z |(\hat{B})_z \cap A \rbrace$$
  \end{center}  
  \begin{itemize}
    \item Grows or thickens objects
  \end{itemize}
\end{block}
\end{frame}

\section{Opening \& Closing}

\subsection{Opening}

\begin{frame}
\frametitle{Morphological Image Processing}
%\framesubtitle{Erosion}
\begin{block}{Opening}\scriptsize
  \begin{center}
    \includegraphics[height=0.2\textheight]{images/opening.png}
    $$A \circ B = (A \ominus B) \oplus B$$
  \end{center}  
  \begin{itemize}
    \item Erosion followed by dilation
    \item Smoothes contour of objects
    \item Breaks narrow strip
    \item Eliminates protusions
  \end{itemize}
\end{block}
\end{frame}

\subsection{Closing}

\begin{frame}
\frametitle{Morphological Image Processing}
%\framesubtitle{Erosion}
\begin{block}{Closing}\scriptsize
  \begin{center}
    \includegraphics[height=0.2\textheight]{images/closing.png}
    $$A \bullet B = (A \oplus B) \ominus B$$
  \end{center}
  \begin{itemize}
    \item Dilation followed by erosion
    \item Smoothes section of contours
    \item Fuses narrow breaks
    \item Eliminate small holes
    \item Fill gaps in the contour
  \end{itemize}
\end{block}
\end{frame}

\section{Hit-or-Miss Transformation}

\begin{frame}
\frametitle{Morphological Image Processing}
%\framesubtitle{Erosion}
\begin{block}{Hit-or-Miss Transformation}\scriptsize
  \begin{center}
    \includegraphics[height=0.5\textheight]{images/hit-miss.png}
    $$A \otimes B = (A \ominus X) \cap \left[A^c (W - X) \right]$$
  \end{center}
  % \begin{itemize}
  %   \item Dilation followed by erosion
  %   \item Smoothes section of contours
  %   \item Fuses narrow breaks
  %   \item Eliminate small holes
  %   \item Fill gaps in the contour
  % \end{itemize}
\end{block}
\end{frame}

\section{Basic Algorithms}

%\subsection{Boundary Extraction}

\begin{frame}
\frametitle{Morphological Image Processing}
%\framesubtitle{Erosion}
\begin{block}{Boundary Extraction}\scriptsize
  \begin{center}
    $$\beta(A) = A - (A \ominus B)$$
  \end{center}
\end{block}
\begin{block}{Hole Filling}\scriptsize
  \begin{center}
    $$X_k = (X_{k-1} \oplus B) \cap A^c \quad k = 1,2,3, \cdots$$
  \end{center}
\end{block}
\begin{block}{Extraction of Connected Components}\scriptsize
  \begin{center}
    $$X_k = (X_{k-1} \oplus B) \cap A \quad k = 1,2,3, \cdots$$
  \end{center}
\end{block}
\end{frame}

\begin{frame}
\frametitle{Morphological Image Processing}
\begin{block}{Convex Hule}\scriptsize
  \begin{center}
    $$X_k = (X_{k-1} \otimes B^i) \cap A^c \quad i = 1,2,3,4 \quad k = 1,2,3, \cdots$$
  \end{center}
\end{block}
\begin{block}{Thinning}\scriptsize
  \begin{center}
    $$A \oslash B = A - (A \otimes B)$$
  \end{center}
\end{block}
\begin{block}{Thickening}\scriptsize
  \begin{center}
    $$A \odot B = A \cup (A \otimes B)$$
  \end{center}
\end{block}
\end{frame}

% \section{Fundamentals}
% \begin{frame}
% \frametitle{Fundamentals}
% \begin{block}{Fourier Series}\scriptsize
%   \begin{itemize}
%     \item Time: Continuous + \emph{Periodic} $\rightarrow$ Frequency: \emph{Digital (coefficients)}
%   \end{itemize}
% \end{block}
% \begin{block}{Fourier Transform}\scriptsize
%   \begin{itemize}
%     \item Time: Continuous + \emph{Non-periodic} $\rightarrow$ Frequency: \emph{Continuous}
%   \end{itemize}
% \end{block}
% \begin{block}{Discrete Time Fourier Transform}\scriptsize
%   \begin{itemize}
%     \item Time: \textbf{Discrete} + \emph{Non-periodic} $\rightarrow$ Frequency: \emph{Continuous} + \textbf{Periodic}
%   \end{itemize}
% \end{block}
% \begin{block}{Discrete Fourier Transform}\scriptsize
%   \begin{itemize}
%     \item Time: \textbf{Discrete} + \emph{Periodic} $\rightarrow$ Frequency: \emph{Digital} + \textbf{Periodic}
%   \end{itemize}
% \end{block}
% \end{frame}


% \section{Fundamentals}
% \begin{frame}
% \frametitle{Fundamentals}
% \begin{block}{The different transforms}
% \centering
% \includegraphics[width = 0.9\textwidth]{images/dft.png}
% \end{block}
% \end{frame}


% %-----------------
% \section{Introduction}

% \subsection{Formulation}

% \begin{frame}
% \frametitle{Introduction to Fourier Transform}
% \begin{block}{Fourier transform of continuous function $f(x)$}
% 	$$\Im\lbrace f(x)\rbrace = F(u) = \int\limits_{-\infty}^{\infty} f(x) \exp(-2 j \pi ux) dx$$
% \begin{itemize}\footnotesize
% \item Integral shows $F(u)$ composed of infinite sum of sine and cosine 
% \item Each $u$ value determines the frequency of its corresponding $\sin$ and $\cos$ pair
% \end{itemize}
% \end{block}
% \end{frame}
% %----------
% \begin{frame}
% \frametitle{Introduction to Fourier Transform}
% \begin{block}{Inverse Fourier transform of $F(u)$}
% 	$$\Im^{-1}\lbrace F(u)\rbrace = f(x) = \int\limits_{-\infty}^{\infty} F(u) \exp(2j\pi ux) du$$
% \end{block}

% \begin{itemize}
% \item The above equations ($\Im\lbrace f(x)\rbrace$ and $\Im^{-1}\lbrace F(u)\rbrace$) represent the Fourier transform pair 
% \end{itemize}
% \end{frame}
% %----------
% \begin{frame}
% \frametitle{Introduction to Fourier Transform}
% \begin{block}{Fourier transform pair for $f(x,y)$ of two variables}
% 	$$\Im\lbrace f(x,y)\rbrace = F(u,v) =\int \int\limits_{-\infty}^{\infty} f(x,y) \exp(-2j\pi (ux+vy)) dx dy$$
% 	$$\Im^{-1}\lbrace F(u,v)\rbrace = f(x,y) =\int \int\limits_{-\infty}^{\infty} F(u,v) \exp(2j\pi (ux+vy)) du dv$$
% \end{block}
% \end{frame}
% %----------
% \begin{frame}	
% \frametitle{Introduction to Fourier Transform}
% \framesubtitle{Discrete Fourier Transform (DFT)}
% \begin{block}{Fourier transform pair - 1D}
% \small{
% 	$$F(u) = \frac{1}{M}\sum^{M-1}_{x=0} f(x) \exp(-2\pi ux/M)$$ \vspace{0.1cm} for $u = 0, 1, 2, ..., M-1$
% 	}
% \small{
% 	$$f(x) = \sum^{M-1}_{u=0} f(u) \exp(2\pi ux/M)$$ \vspace{0.1cm} for $x = 0, 1, 2, ..., M-1$
% 	}		
% \end{block}
% \end{frame}
% %----------
% \begin{frame}	
% \frametitle{Introduction to Fourier Transform}
% \framesubtitle{Discrete Fourier Transform (DFT)}
% \begin{block}{Fourier transform pair - 2D}
% \small{
% 	$$F(u,v) = \frac{1}{MN} \sum^{M-1}_{x=0}\sum^{N-1}_{y=0} f(x,y) \exp(-2\pi (ux/M + uy/N))$$ \vspace{0.1cm} for $u = 0, 1, 2, ..., M-1$, and $v = 0, 1, 2, ..., N-1$ 
% 	}
% \small{
% 	$$f(x,y) = \sum^{M-1}_{u=0} \sum^{N-1}_{v=0} f(u,v) \exp(2\pi (ux/M+uv/N))$$ \vspace{0.1cm} for $x = 0, 1, 2, ..., M-1$ and $y = 0, 1, 2, ..., N-1$
% 	}		
% \end{block}
% \end{frame}
% %----------
% \begin{frame}	
% \frametitle{Introduction to Fourier Transform}
% \framesubtitle{Discrete Fourier Transform (DFT)}
% \begin{block}{Aliasing}
% \centering{
% \includegraphics[width = 0.8\textwidth]{images/sampling_dirac.png}
% }
% \end{block}
% \end{frame}
% %----------
% \begin{frame}	
% \frametitle{Introduction to Fourier Transform}
% \framesubtitle{Discrete Fourier Transform (DFT)}
% \begin{block}{Sampling theorem}
% \begin{itemize}\footnotesize
% \item A continuous function can be \emph{recovered uniquely} from a set of its samples
% \item Under the sampling condition that:
% $$\frac{1}{\Delta T} > 2 \mu_{\text{max}}$$
% \end{itemize}
% \end{block}
% \begin{block}{Fourier transform of sampled function}{\tiny
% $$\widetilde{F}(\mu) = F(\mu) * S(\mu) \ ,$$
% with
% $$S(\mu) = \frac{1}{\Delta T} \sum_{n=-\infty}^{\infty} \delta(\mu - \frac{n}{\Delta T})$$
% $$\widetilde{F}(\mu)=\frac{1}{\Delta T} \sum_{n=-\infty}^{\infty} F(\mu - \frac{n}{\Delta T})$$
% }
% \end{block}

% \end{frame}
% %----------
% \begin{frame}	
% \frametitle{Introduction to Fourier Transform}
% \framesubtitle{Discrete Fourier Transform (DFT)}
% \begin{block}{Aliasing}
% \begin{itemize}
%   \item A band-limited function $f(t,z)$ can be recovered if the sampling intervals are
% $$\Delta T < \frac{1}{2\mu_{\text{max}}}$$
% and 
% $$\Delta Z < \frac{1}{2\nu_{\text{max}}}$$
% \end{itemize}
% \end{block}
% \end{frame}
% %----------
% \begin{frame}	
% \frametitle{Introduction to Fourier Transform}
% \framesubtitle{Discrete Fourier Transform (DFT)}
% \begin{block}{Aliasing}
% \centering{
% \includegraphics[height = 0.6\textheight]{images/aliasing.jpg}
% }
% \end{block}
% \end{frame}
% %----------
% \begin{frame}
% \frametitle{Introduction to Fourier Transform}
% \framesubtitle{Discrete Fourier Transform (DFT)}
% \begin{itemize}
% 	\item The Fourier transform of a real function is generally complex and we use polar coordinates: 
% \end{itemize}
% \begin{columns}
% \column{0.5\textwidth}
% \begin{block}{1D - DFT}
% \scriptsize{
% $$F(u) = R(u)+jI(u)$$
% $$F(u) = \vert F(u) \vert e^{j \phi (u)} $$
% Magnitude spectrum:
% $$\vert F(u) \vert = [R^{2}(u)+ I^{2}(u)]^{0.5}$$
% Phase spectrum:
% $$ \phi (u) = \tan^{-1} \left[ \frac{I(u)}{R(u)} \right] $$
% }
% \end{block}
% \column{0.5\textwidth}
% \begin{block}{2D - DFT}
% \scriptsize{
% $$F(u,v) = R(u,v)+jI(u,v)$$
% $$F(u,v) = \vert F(u,v) \vert e^{j \phi (u,v} $$
% Magnitude spectrum:
% $$\vert F(u) \vert = [R^{2}(u,v)+ I^{2}(u,v)]^{0.5}$$
% Phase spectrum:
% $$ \phi (u,v) = \tan^{-1} \left[ \frac{I(u,v)}{R(u,v)} \right] $$
% }
% \end{block}
% \end{columns}			
% \end{frame}
% %----------
% \begin{frame}
% \frametitle{Introduction to Fourier Transform}
% \framesubtitle{Discrete Fourier Transform (DFT)}
% \begin{block}{Phase and Magnitude Spectrum}
% \begin{figure}
%   \centering
%   \subfloat[Original]{\includegraphics[width=.25\textwidth]{./images/rect1.png}}\qquad
%   \subfloat[Magnitude]{\includegraphics[width=.25\textwidth]{./images/rect1_mod.png}}\qquad
%   \subfloat[Phase]{\includegraphics[width=.25\textwidth]{./images/rect1_phase.png}}
% \end{figure}
% % \includegraphics[width=.25\textwidth]{./images/rect1.png}
% % \includegraphics[width=.25\textwidth]{./images/rect1_mod.png}~
% % \includegraphics[width=.25\textwidth]{./images/rect1_phase.png}
% \end{block}
% \end{frame}
% %----------
% \subsection{DFT Properties}
% \begin{frame}
% \frametitle{Introduction to Fourier Transform}
% \framesubtitle{DFT properties}
% \begin{block}{Effect of Translation in Spatial Domain}
% %\begin{itemize}
% $$ f(x-x{0}, y-y_{0}) \Leftrightarrow F(u,v)e^{-j2\pi(\frac{x_{0}u}{M} + \frac{y_{0}v}{N})}$$
% $$ f(u-u{0}, v-v_{0}) \Leftrightarrow f(x,y)e^{j2\pi(\frac{u_{0}x}{M} + \frac{v_{0}y}{N})}$$
% %\centering{
% %\includegraphics[scale = 0.2]{images/F1_ex2_OriMagPhase_translation.png}
% %\item A shift in spatial domain, does not affect the magnitude in frequency domain  
% %}
% %\end{itemize}
% \end{block}
% \begin{block}{Periodicity}
% \begin{itemize}
% \item The DFT and its inverse are periodic with period $N$
% \end{itemize}
% $$F(u,v) = F(u+M,v) = F(u, v+N) = F(u+M, v+N)$$ 
% \vspace{-5mm}
% \begin{itemize}
% \item Due to the periodicity, we can shift the spectrum such that $f(x)e^{jpix + y} = f(x,y)(-1)^{x+y}$. It is corresponding to the function \emph{fftshift}.
% \end{itemize}
% \end{block}
% \end{frame}
% %----------
% \begin{frame}
% \frametitle{Introduction to Fourier Transform}
% \framesubtitle{Discrete Fourier Transform (DFT)}
% \begin{block}{Phase and Magnitude Spectrum}
% \begin{figure}
%   \centering
%   \subfloat[Original]{\includegraphics[width=.25\textwidth]{./images/rect2.png}}\qquad
%   \subfloat[Magnitude]{\includegraphics[width=.25\textwidth]{./images/rect2_mod.png}}\qquad
%   \subfloat[Phase]{\includegraphics[width=.25\textwidth]{./images/rect2_phase.png}}
% \end{figure}
% % \includegraphics[width=.25\textwidth]{./images/rect1.png}
% % \includegraphics[width=.25\textwidth]{./images/rect1_mod.png}~
% % \includegraphics[width=.25\textwidth]{./images/rect1_phase.png}
% \end{block}
% \end{frame}
% %----------
% \begin{frame}
% \frametitle{Introduction to Fourier Transform}
% \framesubtitle{Discrete Fourier Transform (DFT)}
% \begin{block}{2D- DFT - Basic properties}
% \begin{itemize}
% \scriptsize{
% 	\item Power spectrum: 
% 	$$P(u,v) = \vert F(u,v)^2 \vert $$
% 	\item Average gray level of the image: 
% 	$$F(0,0) = \frac{1}{MN}\sum^{M-1}_{x=0}\sum^{N-1}_{y=0} f(x,y)$$
% 	\item Symmetric spectrum: 
% 	$$F(u,v) = F*(-u,-v) $$
% 	$$\vert F(u,v) \vert  = \vert F(-u,-v) \vert $$
% }
% \end{itemize}
% \end{block}
% \end{frame}

% %----------
% \begin{frame}
% \frametitle{Introduction to Fourier Transform}
% \framesubtitle{Discrete Fourier Transform (DFT)}
% \begin{block}{Magnitude and phase interpretation}
%   \begin{itemize}\footnotesize
%   \item Magnitude spectrum tells the amplitude of the sinusoids that forms the image
%   \item For any given frequency, large amplitude indicates high influence of that frequency, while the low amplitude indicate the opposite 
%   \item Phase indicate the displacement of the sinusoids with respect to their origin 
%   \end{itemize}
%   \begin{center}
%     \includegraphics[scale= 0.3]{images/F1_ex_original.png}\
%     \includegraphics[scale= 0.3]{images/F1_ex_FS.png}
%   \end{center} 
% \end{block}
% \end{frame}
% %----------
% \begin{frame}
% \frametitle{Introduction to Fourier Transform}
% \framesubtitle{Discrete Fourier Transform (DFT)}
% \begin{block}{Information in magnitude spectrum}
% \begin{itemize}
%   \item Cancelling the magnitude remove information about the pixel intensities in the spatial domain
%   \item See notebook
% \end{itemize}
% \end{block}
% \begin{block}{Information in phase spectrum}
% \begin{itemize}
%   \item Cancelling the phase remove information about the spatial information in the spatial domain
%   \item See notebook
% \end{itemize}
% \end{block}
% \end{frame}
% %----------
% \begin{frame}
% \frametitle{Introduction to Fourier Transform}
% \framesubtitle{Discrete Fourier Transform (DFT)}
% \begin{block}{Relationship between spatial and frequency intervals}
% \begin{center}
%   Spatial and frequency are inversely proportional such that:
% \end{center}
% $$\Delta u = \frac{1}{M \Delta T}$$
% and
% $$\Delta v = \frac{1}{N \Delta Z}$$
% \end{block}
% \end{frame}
% \begin{frame}
% \frametitle{Introduction to Fourier Transform}
% \framesubtitle{Discrete Fourier Transform (DFT)}
% \begin{block}{Phase and Magnitude Spectrum}
% \begin{figure}
%   \centering
%   \subfloat[Original]{\includegraphics[width=.25\textwidth]{./images/rect2.png}}\qquad
%   \subfloat[Magnitude]{\includegraphics[width=.25\textwidth]{./images/rect2_mod.png}}\qquad
%   \subfloat[Phase]{\includegraphics[width=.25\textwidth]{./images/rect2_phase.png}}
% \end{figure}
% % \includegraphics[width=.25\textwidth]{./images/rect1.png}
% % \includegraphics[width=.25\textwidth]{./images/rect1_mod.png}~
% % \includegraphics[width=.25\textwidth]{./images/rect1_phase.png}
% \end{block}
% \end{frame}
% %----------
% \begin{frame}
% \frametitle{Introduction to Fourier Transform}
% \framesubtitle{DFT Properties}
% \begin{block}{Effect of rotation in Spatial Domain}
% \scriptsize{
% $$f(x,y)\Leftrightarrow f(r, \theta)$$ 
% $$F(u,v)\Leftrightarrow F( w, \varphi)$$
% $$f(r, \theta+\theta_{0}) \Leftrightarrow F(w,\varphi+\theta)$$}
% \begin{itemize}
% \scriptsize{
% \item Rotating $f(x,y)$ by $\theta$ rotates $F(u,v)$ by the same angle and vice versa.
% }
% \end{itemize}
% \end{block}
% \begin{block}{Effect of Translation in Spatial Domain}\scriptsize
% \scriptsize{
% $$ f(x-x{0}, y-y_{0}) \Leftrightarrow F(u,v)e^{-j2\pi(\frac{x_{0}u}{M} + \frac{y_{0}v}{N})}$$
% $$ f(u-u{0}, v-v_{0}) \Leftrightarrow f(x,y)e^{j2\pi(\frac{u_{0}x}{M} + \frac{v_{0}y}{N})}$$}
% \begin{itemize}
% \scriptsize{
% \item No changes in the magnitude.
% }
% \end{itemize}
% \end{block}
% \end{frame}
% %---------
% \begin{frame}
% \frametitle{Introduction to Fourier Transform}
% \framesubtitle{DFT Properties}
% \begin{block}{Distributing and Scaling}
% \scriptsize{
% \begin{itemize}
% \item Distributive over addition but not over multiplication
% $$\Im\lbrace f_{1}(x,y)+f_{2}(x,y)\rbrace = \Im\lbrace f_{1}(x,y) \rbrace +\Im\lbrace f_{2}(x,y)\rbrace $$ 
% $$\Im\lbrace f_{1}(x,y).f_{2}(x,y)\rbrace \neq \Im\lbrace f_{1}(x,y) \rbrace .\Im\lbrace f_{2}(x,y)\rbrace $$ 
% \item For two scalars a and b 
% $$ af(x,y)\Leftrightarrow aF(u,v) $$ 
% $$ f(ax,by) \Leftrightarrow \frac{1}{\vert ab \vert} F(u/a, v/b)$$ 
% \end{itemize}
% }
% \end{block}
% \end{frame}
% %---------
% \begin{frame}
% \frametitle{Introduction to Fourier Transform}
% \framesubtitle{DFT Properties}
% \begin{block}{Conjugate Symmetry}
% \tiny{
% \begin{itemize}
% \item Conjugate symmetry
% \item[] $$ F(u,v) = F*(-u,-v)$$ 
% \item[] $$ \vert F(u,v) \vert = \vert F(-u,-v)\vert $$
% \end{itemize}
% }
% \end{block}
% \begin{block}{Separability}
% \tiny{
% \begin{itemize}
% \item DFT pair can be expressed in separable forms: 
% $$F(u,v) = \frac{1}{M}\sum^{M-1}_{x=0} F(x,v) exp[-j2\pi ux/M]$$
% $$F(x,v) = \left[ \frac{1}{N}\sum^{N-1}_{y =0} f(x,y)exp[-2\pi vy/N]\right]$$
% \end{itemize}}
% \end{block}
% \end{frame}
% %---------
% \begin{frame}
% \frametitle{Introduction to Fourier Transform}
% \framesubtitle{DFT Properties}
% \begin{block}{Separability}
% \scriptsize{
% \begin{itemize}
% \item For each value of $x$, the expression inside the brackets is a 1-D transform
% \item 2-D $F(x,v)$ is obtained by taking a transform along each row of $f(x,y)$ and multiplying the result by N 
% \item $F(u,v)$ is obtained by making a transform along each column of $F(x,v)$
% \item[]
% \centering{
% \includegraphics[scale=0.3]{images/F1_Separability}}
% \end{itemize}
% }
% \end{block}
% \end{frame}
% %---------
% \begin{frame}
% \frametitle{Introduction to Fourier Transform}
% \framesubtitle{DFT Properties}
% \begin{block}{Convolution}
% $$f(x,y) * h(x,y) = \sum_{m=0}^{M-1} \sum_{n=0}^{N-1} f(m,n) h(x-m, y-n)$$
% $$f(x,y) * h(x,y) \Longleftrightarrow F(u,v) H(u,v)$$
% \end{block}
% \end{frame}
% %----------
% % \begin{frame}
% % \frametitle{Introduction to Fourier Transform}
% % \framesubtitle{DFT Properties}
% % \begin{block}{Correlation}
% % \begin{itemize}
% % \item Convolution theorem with FT pair: 
% % $$f(x,y) \circ g(x,y) \Leftrightarrow F^\ast(u,v)G(u,v)$$
% % $$f^\ast(x,y)g(x,y) \Leftrightarrow F(u,v) \circ G(u,v)$$
% % \item Discrete equivalent: 
% % $$f_{e}(x)\circ g_{e}(x) = \frac{1}{M} \sum^{M-1}_{m=0}f_{e}^{\ast}(m)g_{e}(x-m)$$ 
% % \end{itemize}
% % \end{block}
% % \end{frame}
% %----------
% \section{Image Enhancement}
% \subsection{Introduction}
% \begin{frame}
% \frametitle{Image Enhancement}
% \framesubtitle{Basic Filtering in Frequency Domain}
% \begin{itemize}
% 	\item[]
% 	\begin{center}
% 	\includegraphics[width = 0.6\textwidth, height = 0.22\textheight]{images/F1_steps.png}
% 	\end{center}
% 	\scriptsize{
% 	\item Compute Fourier transform of image $F(u,v)$
% 	\item Multiply the result by a filter transfer function $H(u,v)$
% 	\item Take the inverse transform to produce the enhanced image 
% 	$$G(u,v) = H(u,v) F(u,v)$$
% 	$$g(x,y) = \Im^{-1}[G(u,v)] $$
% 	}
% \end{itemize}
% \scriptsize{	
% \begin{block}{Attention!!}
% Because of periodicity when taking DFT we have to avoid wraparound error or aliasing
% \end{block}}
% \end{frame}
% %----------
% % \begin{frame}
% % \frametitle{Image Enhancement}
% % \begin{block}{2D- DFT example}
% % \begin{center}
% % \includegraphics[scale= 0.3]{images/F1_ex_original.png}\
% % \includegraphics[scale= 0.3]{images/F1_ex_FS.png}
% % \end{center} 
% % \scriptsize{SEM image of a damaged integrated circuit}  \quad \quad \scriptsize{Fourier spectrum}
% % \end{block}
% % \end{frame}
% %----------
% \begin{frame}
% \frametitle{Image Enhancement}
% \framesubtitle{Zero-padding}
% \centering{
% \includegraphics[height = 0.65\textheight]{images/zeropadding.png}
% }
% \end{frame}
% %----------
% \begin{frame}
% \frametitle{Image Enhancement}
% \framesubtitle{Filtering}
% \begin{block}{Notch filter}\footnotesize
% \begin{itemize}
% 	\item It forces the average of the image to be 0 
% 	\item F(0,0) = 0 and then take the inverse 
% 	\[
%  	H(u,v) = 
%   	\begin{cases} 
%    	0 & \text{if } (u,v) = M/2, N/2 \\
%    	1 & \text{otherwise}
%   	\end{cases}
% 	\]
% 	\item []
% 	\begin{center}
% 	\includegraphics[scale=0.2]{images/F1_ex_notch.png}
% 	\end{center}
% \end{itemize}
% \end{block}
% \end{frame}
% %---------
% \begin{frame}
% \frametitle{Image Enhancement}
% \framesubtitle{Filtering}
% \begin{block}{Low-pass filter}
% \begin{itemize}
% 	\item Reduces the high frequency contents (blurring or smoothing)
% 	\item []
% 	\begin{center}
% 	\includegraphics[scale=0.24]{images/F1_ex_LP.png}
% 	\end{center}
% \end{itemize}
% \end{block}

% \end{frame}
% %---------
% \begin{frame}
% \frametitle{Image Enhancement}
% \framesubtitle{Filtering}
% \begin{block}{High-pass filter}
% \begin{itemize}
% 	\item Increase the magnitude of the high frequency components relative to low frequency components (sharpening)
% 	\item []
% 	\begin{center}
% 	\includegraphics[scale=0.24]{images/F1_ex_HP.png}
% 	\end{center}
% \end{itemize}
% \end{block}
% \end{frame}
% %---------
% \subsection{Low-pass filter (Smoothing)}
% \begin{frame}
% \frametitle{Image Enhancement}
% \framesubtitle{Low-pass filter (Smoothing)}
% \begin{itemize}
% \item Edges, noise contribute significantly to the high-frequency content of the FT of an image
% \item Blurring/smoothing is achieved by reducing a specified range of high-frequency components
% $$ G(u,v) = H(u,v)F(u,v)$$
% \end{itemize}
% \begin{block}{Different Types}
% \begin{itemize}
% \item Ideal 
% \item Butterworth
% \item Gaussian
% \end{itemize}
% \end{block}
% \end{frame}
% %--------
% \begin{frame}
% \frametitle{Image Enhancement}
% \framesubtitle{Low-pass filter}
% \begin{block}{Ideal low-pass filter}
% \scriptsize{
% \begin{itemize}
% \item[] 	\[
%  	H(u,v) = 
%   	\begin{cases} 
%    1 & \text{if } D(u,v) \leq D_{0} \\
%    0 & \text{if } D(u,v) > D_{0}
%   	\end{cases}
% 	\]
% 	\noindent $D_{0}$ is a specified non-negative quantity (Cutoff frequency) \\
% 	D(u,v) is the distance from point $(u,v)$ to the center of frequency rectangle\\
% 	$F_{c}(u,v) = (M/2, N/2)$\\
% 	$D(u,v) = (u^2+v^2)^{1/2}$\\
% 	\centering{
% 	\includegraphics[scale=0.25]{images/F1-LP-Ideal.png}}
% \end{itemize}
% }
% \end{block}
% \end{frame}
% %---------
% \begin{frame}
% \frametitle{Image Enhancement}
% \framesubtitle{Low-pass filter}
% \begin{block}{Ideal low-pass filter}
% \scriptsize{Original image and results of ideal low-pass with cutoff frequencies $\{$5,15,30,80,230$\}$}
% \centering{
% \includegraphics[width= 0.5\textheight, height= 0.6\textheight]{images/F1-ex-LP-ideal.png}
% }
% \end{block}
% \end{frame}
% %---------
% \begin{frame}
% \frametitle{Image Enhancement}
% \framesubtitle{Low-pass filter}
% \begin{block}{Butterworth low-pass filter}
% \scriptsize{
% \begin{itemize}
% \item[] 	$$ H(u,v) = \frac{1}{1+[D(u,v)/D_{0}]^{2n}}$$

% \begin{itemize}
% \scriptsize{
% 	\item $n$ is the order of the filter 
% 	\item $D_{0}$ cutoff frequency locus (distance from the origin)
% 	\item $D(u,v) = (u^2 + v^2)^{1/2}$ filter characteristics
% 	}
% \end{itemize}
% \item Does not have a sharp discontinuity
% \item Does not establish a clear cutoff between passed and filtered frequencies
% \item When $D(u,v) = D_{0}$, $H(u,v) = 0.5$\\
% 	\centering{
% 	\includegraphics[scale=0.23]{images/F1-LP-BW.png}}	
% \end{itemize}
% }
% \end{block}
% \end{frame}
% %---------
% \begin{frame}
% \frametitle{Image Enhancement}
% \framesubtitle{Low-pass filter}
% \begin{block}{Butterworth low-pass filter}
% \scriptsize{Original image and results of Butterworth low-pass with order $2$ and cutoff frequencies $\{$5,15,30,80,230$\}$}\\
% \centering{
% \includegraphics[width= 0.5\textheight, height= 0.6\textheight]{images/F1-ex-LP-BW.png}
% }
% \end{block}
% \end{frame}
% %---------
% \begin{frame}
% \frametitle{Image Enhancement}
% \framesubtitle{Low-pass filter}
% \begin{block}{Butterworth filter}
% \scriptsize{Spatial representation of Butterworth low-pass with orders of $\{$1,2,5,20$\}$}
% \centering{
% \includegraphics[width= 1.0\textheight, height= 0.5\textheight]{images/F1-ex-LP-BW2.png}
% }
% \end{block}
% \end{frame}
% %---------
% \begin{frame}
% \frametitle{Image Enhancement}
% \framesubtitle{Low-pass filter}
% \begin{block}{Gaussian low-pass filter}

% \begin{itemize}
% \item[] 	$$ H(u,v) = e^{-D^{2}(u,v)/2\sigma^{2}}$$ 
% \begin{itemize}
% 	\item $\sigma = D_{0}$ cutoff frequency 
% 	\item $D(u,v) = (u^2 + v^2)^{1/2}$ Distance from the FT center
% \end{itemize}
% \item The inverse FT of Gaussian is also a Gaussian\\
% 	\centering{
% 	\includegraphics[scale=0.23]{images/F1-LP-Gaussian.png}}	
% \end{itemize}

% \end{block}
% \end{frame}
% %----------
% \begin{frame}
% \frametitle{Image Enhancement}
% \framesubtitle{Low-pass filter}
% \begin{block}{Gaussian low-pass filter}
% \scriptsize{Original image and results of Gaussian low-pass with cutoff frequencies $\{$5,15,30,80,230$\}$}\\
% \centering{
% \includegraphics[width= 0.5\textheight, height= 0.6\textheight]{images/F1-ex-LP-Gaussian.png}
% }
% \end{block}
% \end{frame}
% %----------
% \subsection{High-pass filter (Sharpening)}
% \begin{frame}
% \frametitle{Image Enhancement}
% \framesubtitle{High-pass filter (Sharpening)}
% \begin{itemize}
% 	\item Attenuating low-frequency components without disturbing high-frequency information.
% 	\item [] $$H_{hp}(u,v) = 1 - H_{lp}(u,v)$$
% \end{itemize}
% \begin{block}{Different Types}
% \begin{itemize}
% 	\item Ideal 
% 	\item Butterworth
% 	\item Gaussian
% \end{itemize}
% \end{block}
% \end{frame}
% %-------------
% \begin{frame}
% \frametitle{Image Enhancement}
% \framesubtitle{High-pass filter}
% \begin{block}{Ideal high-pass filter}
% \scriptsize{
% \begin{itemize}
% \item[] 	\[
%  	H(u,v) = 
%   	\begin{cases} 
%    0 & \text{if } D(u,v) \leq D_{0} \\
%    1 & \text{if } D(u,v) > D_{0}
%   	\end{cases}
% 	\]
% \item Opposite of the ideal low-pass\\
% 	\centering{
% 	\includegraphics[scale=0.4]{images/F1-HP-Ideal.png}}
% \end{itemize}
% }
% \end{block}
% \end{frame}
% %-------------
% \begin{frame}
% \frametitle{Image Enhancement}
% \framesubtitle{High-pass filter}
% \begin{block}{Butterworth high-pass filter}
% \scriptsize{
% \begin{itemize}
% \item[] 	$$ H(u,v) = \frac{1}{1+[D_{0}/D(u,v)]^{2n}}$$
% \item[]
% 	\centering{
% 	\includegraphics[scale=0.4]{images/F1-HP-BW.png}}	
% \end{itemize}
% }
% \end{block}
% \end{frame}
% %--------------
% \begin{frame}
% \frametitle{Image Enhancement}
% \framesubtitle{High-pass filter}
% \begin{block}{Gaussian high-pass filter}

% \begin{itemize}
% 	\item[] 	$$ H(u,v) = 1 - e^{-D^{2}(u,v)/2\sigma^{2}}$$ 
% 	\item[]
% 	\centering{
% 	\includegraphics[scale=0.4]{images/F1-HP-Gaussian.png}}	
% \end{itemize}
% \end{block}
% \end{frame}
% %----------
% \begin{frame}
% \frametitle{Image Enhancement}
% \framesubtitle{High-pass filter}
% \begin{block}{Example of Ideal, Butterworth and Gaussian high-pass}

% \begin{itemize}
% 	\item[] \scriptsize{$D_{0} =$ $\{$30, 60, 160 $\}$}
% 	\item[] \centering{\scriptsize{Ideal High-pass}} \\
% 	\centering{
% 	\includegraphics[scale=0.14]{images/F1-ex-HP-ideal.png}}
% 	\item[] \scriptsize{Butterworth High-pass, $n =2$} \\
% 	\centering{
% 	\includegraphics[scale=0.14]{images/F1-ex-HP-BW.png}}	
% 	\item[] \scriptsize{Gaussian High-pass} \\
% 	\centering{
% 	\includegraphics[scale=0.14]{images/F1-ex-HP-Gaussian.png}}		
% \end{itemize}
% \end{block}
% \end{frame}

% %-----------
% \begin{frame}
% \frametitle{Image Enhancement}
% \framesubtitle{High-pass filter}
% \begin{block}{Recall}
% \begin{itemize}
% \scriptsize{
% 	\item[] 	$$ \nabla ^2 f = \frac{\partial^2 f}{\partial x^2} + \frac{\partial^2 f}{\partial y^2}$$ 
% 	\item[] $$ \nabla ^2 f = \left[f(x+1,y)+f(x-1,y)+f(x,y+1)+f(x,y-1)\right]-4f(x,y)$$ 
% 	}
% \end{itemize}
% \end{block}
% \begin{block}{Laplacian in FD}
% \begin{itemize}
% \scriptsize{
% 	\item[]$$ \Im [\nabla^2 f(x,y)] = -(u^2 +v^2)F(u,v) $$	
% 	\item[]$\rightarrow$ The Laplacian can be implemented in FD by using a filter 
% 	\item[] $$ H(u,v) = -4\pi(u^2 +v^2) $$ 
% 	\item FT pair: 
% 	$$ \nabla^2 f(x,y) \Leftrightarrow [(u-M/2)^2 (v-N/2)^2]F(u,v )$$
% 	}
% \end{itemize}
% \end{block}
% \end{frame}
% %-----------

\end{document}

% %\begin{itemize}
% %\item Also use neighborhood but do not use coefficients 
% %\begin{itemize}
% %\item median filter for noise reduction
% %\end{itemize} 
% %\end{itemize}
